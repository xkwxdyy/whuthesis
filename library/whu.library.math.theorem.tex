% https://tex.stackexchange.com/a/130655/246645
% amsthm 要在 amsmath 后面载入,否则 \qedhere 会在 align* 中失效
\WHULibraryDelayedUntil*{amsmath}
\WHUDependency { package = { amsthm, thmtools } }
\WHUProvideExplLibrary{math.theorem}{\whu@date}{\whu@version}{define math theorem environment}

% 定理类环境
% 去掉.号,但只能通过新定义 style 的方式进行
\declaretheoremstyle
  [
    headpunct     = {},
    postheadspace = { 0.5em },
    headindent    = 2em,
    headfont      = \heiti
  ]
  { whustyle }
\cs_new:Npn \__whu_declare_theorem_with_counter_within:n #1
  {
    \declaretheorem
      [
        style = whustyle,
        name =  \clist_item:nn {#1} {1} ,
        refname = \clist_item:nn {#1} {2} ,
        within = \clist_item:nn {#1} {3} ,
      ]
      { \clist_item:nn {#1} {4} }
  }
\cs_new:Npn \__whu_declare_theorem_with_counter_sibling:n #1
  {
    \declaretheorem
      [
        style   = whustyle ,
        name    = \clist_item:nn {#1} {1} ,
        refname = \clist_item:nn {#1} {2} ,
        sibling = \clist_item:nn {#1} {3} ,
      ]
      { \clist_item:nn {#1} {4} }
  }


\clist_map_function:nN
  {
    { 定理, 定理, section, theorem },
    { 例, 例, section, example },
    { 问题, 问题, section, question },
    { 定义, 定义, section, definition },
    { 性质, 性质, section, property },
    { 命题, 命题, section, proposition },
    { 推论, 推论, section, corollary },
    { 引理, 引理, section, lemma },
    { 公理, 公理, section, axiom },
    { 反例, 反例, section, counterexample },
    { 猜想, 猜想, section, conjecture },
    { 断言, 断言, section, claim },
    { 注, 注, section, remark }
  }
  \__whu_declare_theorem_with_counter_within:n


% 重定义 proof 环境的样式
\RenewDocumentEnvironment { proof } { O{\proofname} +b }
  {
    \par
    \pushQED { \qed }
    \normalfont \topsep6 \p@ \@plus6 \p@ \relax
    \trivlist
    \item \relax
    \group_begin:
      \hspace*{2em}
      \heiti #1
    \group_end:
    \hspace{1em} \ignorespaces
    #2
  }
  {
    \popQED \endtrivlist \@endpefalse
  }