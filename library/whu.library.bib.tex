\WHUProvideExplLibrary{bib}{\whu@date}{\whu@version}{bib configuration}

\whu_after_class:n
  {
    \RequirePackage
      [
        backend      = biber,
        bibstyle     = gb7714-2015,
        sorting      = nty,
        gbalign      = right,
        citestyle    = gb7714-WHU,
        gbnamefmt    = givenahead,
        gbpunctin    = false
      ]{biblatex}
    % > biblatex-gb7714.pdf
    % > 注意:在同一处引用多篇序号连续的文献时,标注标签默认是从两篇文献开始压 缩的,比如同时连续引用两篇和三篇文献时标注分别为 [1-2] 和 [1-3]。若需要修改 从三篇文献开始压缩,比如:两篇和三篇文献分别标注为 [1,2] 和 [1-3],则只需要在 导言区设置计数器 gbrefcompress 的值为 3,即:\setcounter{gbrefcompress}{3}。
    \setcounter { gbrefcompress } { 3 }
    \DefineBibliographyStrings{english}{in={}}
    \DefineBibliographyStrings{english}{incn={}}

    % \AddToHook { cmd / parencite / before }
    %   {
    %     % 调整多个 citation 之间的间隔符 \multicitedelim 为逗号加一个空格
    %     \RenewDocumentCommand \multicitedelim { } { \addcomma\addthinspace }  % 太窄则用 \addspace
    %   }
    % \AddToHook { cmd / cite / before }
    %   {
    %     % https://github.com/hushidong/biblatex-gb7714-2015/issues/137#issuecomment-2041292298
    %     % zepinglee 建议 \cite 的不加空格
    %     \RenewDocumentCommand \multicitedelim { } { \addcomma }
    %   }

    % https://github.com/hushidong/biblatex-gb7714-2015/issues/137#issuecomment-2041298845
    % 改为 biblatex 的处理机制,不用 hook

    % TDOO: 把 \addcomma\addthinspace 做个接口,比如 \__whu_bib_biblatex_parencite_multicitedelimiter:
    \DeclareDelimFormat [ parencite ] { multicitedelimiter } { \addcomma\addthinspace }  % 太窄则用addspace
    % zepinglee 建议 \cite 的不加空格
    \DeclareDelimFormat [ cite ] { multicitedelimiter } { \addcomma }
    \RenewDocumentCommand \multicitedelim { } { \printdelim { multicitedelimiter } }

    \NewCommandCopy \whuprintbibliography \printbibliography
    \RenewDocumentCommand \printbibliography { }
      {
        \sloppy
        \__whu_bib_printbibliography_preset:
        \whuprintbibliography [ heading = bibintoc ]
      }
  }

\cs_new:Nn \__whu_bib_printbibliography_preset: { }