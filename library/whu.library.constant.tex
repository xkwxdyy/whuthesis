\WHUProvideExplLibrary{constant}{\whu@date}{\whu@version}{constant tokens}



% ------------------------------------------ %
% 用于章节标题的 \g__whu_title_constant_..._tl %
% ------------------------------------------ %

% https://github.com/CTeX-org/forum/issues/309#issuecomment-2030804140
% by @muzimuzhi
\cs_new:Nn \__whu_set_title_constant_family:n  % 不能用 protected 
  {
    \keys_set:nn { whu/constant/title } {#1}
  }
\cs_new_protected:Nn \__whu_define_title_constant_family:n
  {
    \keys_define:nn { whu/constant/title }
      {
        #1/zh     .code:n =
          { \__whu_set_title_constant:nnn {#1} { \l_keys_key_str } {##1} }
      , #1/zh-toc .code:n =
          { \__whu_set_title_constant:nnn {#1} { \l_keys_key_str } {##1} }
      , #1/en     .code:n =
          { \__whu_set_title_constant:nnn {#1} { \l_keys_key_str } {##1} }
      , #1/en-toc .code:n =
          { \__whu_set_title_constant:nnn {#1} { \l_keys_key_str } {##1} }
      , #1        .code:n = 
          {
            \keys_set:nn { whu/constant/title/#1 } { ##1 }
            % 没写 zh-toc 的话,就默认和 zh 相同
            \bool_lazy_and:nnT
              { \__whu_title_constant_if_exist_p:nn {#1} {zh} }
              { ! \__whu_title_constant_if_exist_p:nn {#1} {zh-toc} }
              {
                \__whu_set_title_constant:nnn {#1} {zh-toc}
                  { \__whu_use_title_constant:nn {#1} {zh} }
              }
            % 没写 en-toc 的话,就默认和 en 相同
            \bool_lazy_and:nnT
              { \__whu_title_constant_if_exist_p:nn {#1} {en} }
              { ! \__whu_title_constant_if_exist_p:nn {#1} {en-toc} }
              {
                \__whu_set_title_constant:nnn {#1} {en-toc}
                  { \__whu_use_title_constant:nn {#1} {en} }
              }
          }
      }
  }
\cs_new_protected:Nn \__whu_set_title_constant:nnn
  {
    % \tl_const:cn { c__whu_constant_ #1/#2 _tl } {#3}
    % @muzimuzhi: 建议改为变量类型
    \tl_if_exist:cF { g__whu_title_constant_ #1/#2 _tl }
      { \tl_new:c { g__whu_title_constant_ #1/#2 _tl } }
    \tl_set:cn { g__whu_title_constant_ #1/#2 _tl } {#3}
  }

% https://github.com/Sophanatprime/cus/issues/17
% 在写入目录文件、标签时,\protected 的宏不会被展开。\__whu_use_title_constant:nn 不能定义成 \protected。
% > 一个命令能否展开、会不会展开、怎么展开和它所处的上下文有关。 interface3.pdf 里标记 ⋆ 的是可以用于 `e` 和 `f` 参数,标记 ✩ 可以用于 `e` 参数,用在这些参数中它们被完全展开。 写入目录文件、标签、书签等,和用在 `e` 参数中类似,不过用 `\DeclareRobustCommand` 定义的命令也不会展开,所以 `\textbf` 这些可以安全的写入目录文件。
% > `\protected` 命令不会在 `e` 参数中展开,但出现在控制序列的名称中会展开(所以 `\__whu_get_title_constant_csname:nn` 能生效)。 所以使用 `\cs_new:Npn \__whu_use_title_constant:nn` 会在 `e` 参数中展开为 `\tl_use:c { g__whu_title_constant_ #1/#2 _tl }`,`\tl_use:c` 还可以展开,但它先把后面那个变成一个控制序列,变成了 `\tl_use:N <...>`,`\tl_use:N` 还可以展开,它的作用是展开为那个 tl 的内容,而这个 tl 的内容并没有 LaTeX3 名称的命令,所以能够正确读取。
\cs_new:Nn \__whu_use_title_constant:nn
  {
    \tl_if_exist:cT { g__whu_title_constant_ #1/#2 _tl }
      { \tl_use:c { g__whu_title_constant_ #1/#2 _tl } }
  }
\cs_new_protected:Nn \__whu_get_title_constant_csname:nn
  {
    g__whu_title_constant_ #1/#2 _tl
  }
\prg_new_conditional:Nnn \__whu_title_constant_if_exist:nn { p , T , F , TF }
  {
    \tl_if_exist:cTF { \__whu_get_title_constant_csname:nn {#1} {#2} }
      \prg_return_true:
      \prg_return_false:
  }



% ---------------------------------------- %
% 用于封面信息的 \g__whu_information_constant_..._tl %
% -----------------------------------------%

\cs_new:Nn \__whu_set_information_constant_family:n  % 不能用 protected 
  {
    \keys_set:nn { whu/constant/information } {#1}
  }
\cs_new_protected:Nn \__whu_define_information_constant_family:n
  {
    \keys_define:nn { whu/constant/information }
      {
        #1/zh     .code:n =
          { \__whu_set_information_constant:nnn {#1} { \l_keys_key_str } {##1} }
      , #1/en     .code:n =
          { \__whu_set_information_constant:nnn {#1} { \l_keys_key_str } {##1} }
      , #1        .code:n = 
          { \keys_set:nn { whu/constant/information/#1 } { ##1 } }
      }
  }
\cs_new_protected:Nn \__whu_set_information_constant:nnn
  {
    \tl_if_exist:cF { g__whu_information_constant_ #1/#2 _tl }
      { \tl_new:c { g__whu_information_constant_ #1/#2 _tl } }
    \tl_set:cn { g__whu_information_constant_ #1/#2 _tl } {#3}
  }
\cs_new:Nn \__whu_use_information_constant:nn
  {
    \tl_if_exist:cT { g__whu_information_constant_ #1/#2 _tl }
      { \tl_use:c { g__whu_information_constant_ #1/#2 _tl } }
  }
\cs_new_protected:Nn \__whu_get_information_constant_csname:nn
  {
    g__whu_information_constant_ #1/#2 _tl
  }
\prg_new_conditional:Nnn \__whu_information_constant_if_exist:nn { p , T , F , TF }
  {
    \tl_if_exist:cTF { \__whu_get_information_constant_csname:nn {#1} {#2} }
      \prg_return_true:
      \prg_return_false:
  }



% 初始化一些 constant
\clist_map_function:nN
  {
    abstract,
    keywords,
    publications,
    acknowledgement
  }
  \__whu_define_title_constant_family:n
\clist_map_function:nN
  {
    clc,  % 分类号:《中国图书资料分类法》的类号
    secret-level,  % 密级
    udc,  % 《国际十进制分类法UDC》的类号
    school-id,  % 学校编号,10486 是武汉大学的编号
    author,  % 作者
    student-id,  % 学号
    supervisor,  % 导师
    major,  % 专业
    professional-field,  % 研究方向
    universityname,  % 大学名称
  }
  \__whu_define_information_constant_family:n
\__whu_set_information_constant_family:n
  {
    universityname = 
      {
        zh = 武汉大学,
        en = Wuhan~University
      }
  }