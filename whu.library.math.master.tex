\WHUDependency 
  {
    package = { amsmath },
    library = { constant }
  }
\WHUProvideExplLibrary{math-master}{\whu@date}{\whu@version}{Compatible with the configuration of Huang Zhenhua's template for master}


% -------- %
% 超链接设置 %
% -------- %
\AtEndPreamble
  {
    \RequirePackage { hyperref }
    \hypersetup 
      { 
        pdfencoding = auto,
        citecolor   = magenta,
        linkcolor   = blue,
        hidelinks
      }
  }



% ----------------- %
% 不同文档类型下的设置 %
% ----------------- %
\WHULoadLibrary { type }
\cs_set:Npn \__whu_set_type_for_electronic:
  {
    \PassOptionsToClass 
      { a4paper, twoside } { book }
  }
\cs_set:Npn \__whu_set_type_for_print:
  {
    \PassOptionsToClass 
      { a4paper, twoside } { book }
    \AtEndPreamble { \hypersetup { allcolors = black } }
  }
\cs_set:Npn \__whu_set_type_for_library:
  {
    \PassOptionsToClass 
      { a4paper, twoside, openany } { book }
    \AtEndPreamble { \hypersetup { allcolors = black } }
  }



% ---------- %
% 页面布局设置 %
% ---------- %
% TODO: 页脚距离短了,不知道是哪里的设置问题
\cs_new:Npn \whu_math_set_layout:
  {
    \setuplayout
      {
        papername = a4paper,
        top       = 3truecm,
        bottom    = 2.9truecm,
        left      = 2.8truecm,
        right     = 2.5truecm,
        % includefoot,
        % includehead,
        xetex
      }
  }
\whu@ifmoduleloaded { layout }
  { \whu_math_set_layout: }
  { \whu_after_class:n { \whu_math_set_layout: } }



% ------- %
% 页眉页脚 %
% ------- %
\cs_new:Npn \whu_math_set_headfoot:
  {
    \setpagestyle { frontmatter }
      {
        \setheadrulewidth { 0.5pt }
        \setfootrulewidth { 0pt }
        \setcenterhead
          [ 武汉大学硕士专业学位论文 ] % TODO: \ifsmd 武汉大学硕士专业学位论文\else 武汉大学硕士学位论文
          { 标题 } % TODO
        \sethead [LR] {}
        \setcenterfoot 
          { 
            -\,
            \int_to_Roman:n { \c@page }  % 此处不能用 \thepage, 因为在 \frontmatter 和 \mainmatter 之间是小写罗马数字格式
            \,- 
          }
      }
    \setpagestyle { mainmatter } [ frontmatter ]
      {
        \setcenterfoot { -\, \int_to_arabic:n { \c@page } \,- }
      }
    \AddToHook { cmd/frontmatter/after }
      {
        \usepagestyle { frontmatter } 
        \setuptitle [ chapter ] { pagestyle = frontmatter }
      }
    \AddToHook { cmd/mainmatter/after }
      {
        \usepagestyle { mainmatter } 
        \setuptitle [ chapter ] { pagestyle = mainmatter }
      }
  }
\whu@ifmoduleloaded { layout }
  { \whu_math_set_headfoot: }
  { \whu_after_class:n { \whu_math_set_headfoot: } }



% ---------- %
% 章节标题设置 %
% ---------- %
\cs_new:Npn \whu_math_set_title:
  {
    \setsecnumdepth { 4 }
    \setuptitle [ chapter, section, subsection, subsubsection ]
      {
        format = \bfseries\heiti\raggedright
      }
    \setuponetitle { chapter }
      {
        name       = {,},
        number     = { \arabic { chapter } },
        format+    = { \zihao{-2} },
        afterskip  = 30pt,
        beforeskip = 20pt
      }
    \setuponetitle { section }
      {
        format+  = { \zihao{3} },
        number = { \arabic { chapter }.\arabic { section } },
      }
    \setuponetitle { subsection }
      {
        format+  = { \zihao{4} },
        number = { \arabic { chapter }.\arabic { section }. \arabic { subsection } },
      }
    \setuponetitle { subsubsection }
      {
        format+ = { \zihao{-4}\kaishu },
        number  = { \arabic { chapter }.\arabic { section }. \arabic { subsection }. \arabic { subsubsection } },
      }
    \AddToHook { cmd/frontmatter/after }
      {
        \setuponetitle { chapter } { mode = nonumber }
      }
    \AddToHook { cmd/tableofcontents/before }
      {
        \setuponetitle { chapter }  { mode = starred }
      }
    \AddToHook { cmd/mainmatter/after }
      {
        \setuponetitle { chapter }  { mode = normal }
      }
  }
\whu@ifmoduleloaded { struct }
  { \whu_math_set_title: }
  { \whu_after_class:n { \whu_math_set_title: } }




% ------- %
% 目录设置 %
% ------- %
\cs_new:Npn \whu_math_set_toc:
  {
    % 目录标题
    \AddToHook { cmd/tableofcontents/before }
      { \renewcommand { \contentsname } { 目\qquad 录 } }
    % 目录层级
    \setplaintocdepth { 4 }
    % 目录的页码样式
    \AddToHook { cmd/frontmatter/after }
      { \pagenumbering { Roman } }
    \AddToHook { cmd/mainmatter/after }
      { \pagenumbering { arabic } }
  }
\whu@ifmoduleloaded { struct }
  { \whu_math_set_toc: }
  { \whu_after_class:n { \whu_math_set_toc: } }



% ---------- %
% 图表样式设置 %
% ---------- %
\whu_after_class:n
  {
    \RequirePackage { caption }
    \captionsetup
      {
        font     = small,
        textfont = it
      }
  }



% ------- %
% 数学设置 %
% ------- %
\allowdisplaybreaks



% ------- %
% 摘要设置 %
% ------- %
\WHULoadLibrary { abstract }
\__whu_set_constant_family:n
  {
    abstract =
      {
        zh     = 摘 \qquad 要 ,
        zh-toc = 摘要,
        en     = ABSTRACT,
      },
    keywords =
      {
        zh = 关键词 ,
        en = Key~words
      }
  }
\cs_set:Npn \__whu_keywords_zh_format_set:
  { \heiti\zihao{-4} }
\cs_set:Npn \__whu_abstract_zh_begin: 
  {
    开头
  }
\cs_set:Npn \__whu_abstract_zh_end: 
  {
    结尾
  }



% ---------- %
% 参考文献设置 %
% ---------- %
\whu_after_class:n
  {
    \RequirePackage
      [
        backend      = biber,
        bibstyle     = gb7714-2015,
        sorting      = nty,
        gbalign      = right,
        citestyle    = gb7714-WHU,
        gbnamefmt    = givenahead,
        gbpunctin    = false
      ]{biblatex}
    \DefineBibliographyStrings { english } { in={} }
    \DefineBibliographyStrings { english } { incn={} }

    \DeclareCommandCopy { \whuprintbibliography } { \printbibliography }
    \RenewDocumentCommand { \printbibliography } { }
      {
        \sloppy
        \setupnexttitle { format+ = \large }  % 注意,只有当参考文献成功编译出来才会作用到参考文献的 chapter 上,否则就会作用到后面的比如致谢等章节上。因为 \setupnexttitle 是作用于下一个出现的章节命令上。
        \whuprintbibliography [ heading = bibintoc ]
      }
  }



% ------ %
% 学术成果 %
% ------ %
\WHULoadLibrary { publications }
\cs_set:Npn \__whu_publications_format_set:
  {
    \setuponetitle { chapter } 
      { mode = nonumber }
    \__whu_set_constant_family:n
      {
        publications =
          {
            zh = 攻硕期间发表的科研成果目录 ,
          }
      }
  }
% publications 的设置来自 thuthesis
\whu_after_class:n
  {
    \IfPackageLoadedTF { enumitem }
      {
        \newlist { achievements } { enumerate } { 1 }
        \setlist [ achievements ]
          {
            topsep     = 6bp,
            partopsep  = 0bp,
            itemsep    = 6bp,
            parsep     = 0bp,
            leftmargin = 10mm,
            itemindent = 0pt,
            align      = left,
            label      = [\arabic*],
            resume     = achievements,
          }
      }
      { 
        \RequirePackage { enumitem } 
        \newlist { achievements } { enumerate } { 1 }
        \setlist [ achievements ]
          {
            topsep     = 6bp,
            partopsep  = 0bp,
            itemsep    = 6bp,
            parsep     = 0bp,
            leftmargin = 10mm,
            itemindent = 0pt,
            align      = left,
            label      = [\arabic*],
            resume     = achievements,
          }
      }
  }

\cs_set:Npn \__whu_publications_begin:
  {
    \begin{achievements}
  }
\cs_set:Npn \__whu_publications_end:
  {
    \end{achievements}
  }



% --- %
% 致谢 %
% --- %
\WHULoadLibrary { acknowledgement }
\cs_set:Npn \__whu_acknowledgement_format_set:
  {
    \setuponetitle { chapter } 
      {
        mode    = nonumber,
        % format+ = \ziju{2},  % 是否添加似乎并不影响效果(可能因为只有两个字)
      }
    \__whu_set_constant_family:n
      {
        acknowledgement =
          {
            zh-toc = 致谢 ,
            zh     = 致\qquad 谢
          }
      }
  }




% \newif\iflib\libfalse
% \DeclareOption{forlib}{\libtrue}
% \newif\ifsmd\smdfalse
% \DeclareOption{smd}{\smdtrue}
% \ProcessOptions




% \newtoks\fenleihao      %中图分类号
% \newtoks\bianhao         %学校编号
% \newtoks\UDC             %《国际十进制分类法UDC》的类号
% \newtoks\miji            %密级
% \newtoks\Cmajor          %学科专业中文名
% \newtoks\Emajor          %学科专业英文名
% \newtoks\Especiality      %研究方向
% \newtoks\Cspeciality      %研究方向
% \newtoks\Edate           %英文日期
% \newtoks\Etitle          %英文题目
% \newtoks\Eauthor         %作者英文名
% \newtoks\Esupervisor     %指导教师英文名
% \newtoks\Csupervisor     %指导教师中文名
% \newtoks\Schoolname      %学院英文名
% \newtoks\StudentNumber  %学号,硕士用


% %%%=== 封面 ===%%%
% \def\maketitle{%
%   \null
%   \thispagestyle{empty}%
%   %\vspace*{-2cm}
%   \begin{center}\leavevmode
%     {\fangsong \zihao{4}%
%     \begin{tabular}{lp{2cm}p{5cm}lp{2cm}l}
%       \raisebox{-0.5ex}[0pt]{\makebox[1.5cm][s]{分\hfill 类\hfill 号}} & {}\hfill\raisebox{-0.5ex}[0pt]{\the\fenleihao}\hfill{} &  &
%       \raisebox{-0.5ex}[0pt]{\makebox[1.5cm][s]{密\hfill 级}} & {}\hfill\raisebox{-0.5ex}[0pt]{\the\miji}\hfill{} & \\ \cline{2-2}\cline{5-5}
%       \raisebox{-0.5ex}[0pt]{\makebox[1.5cm][s]{U\hfill D\hfill C} } & {}\hfill\raisebox{-0.5ex}[0pt]{\the\UDC}\hfill{}      &  &
%       \raisebox{-0.5ex}[0pt]{\makebox[1.5cm][s]{编\hfill 号}} & {}\hfill\raisebox{-0.5ex}[0pt]{\the\bianhao}\hfill{}   & \\ \cline{2-2}\cline{5-5}
%     \end{tabular}
%     }
%     \par
%     \vspace*{15mm} %插入空白
%     {\songti \zihao{2} % 武汉大学
%     \includegraphics[width=0.4\textwidth]{wudaname.eps}
%     \\[3mm]   \ifsmd  \ziju{0.5} 硕士专业学位论文\else \ziju{1} 硕士学位论文\fi}\\
%     \vspace{2cm}
%     {\kaishu \zihao{1}  \@title \par}%
% %    \vspace{4cm}
%      \vfill\vfill\vfill
%     {\songti\zihao{4}
%     \ifsmd 
%     \begin{tabular}{cp{5.5cm}c}
%       \raisebox{-3ex}[0pt]{\makebox[4.8cm][s]{研\hfill 究\hfill 生\hfill 姓\hfill 名: }} & {\songti {}\raisebox{-3ex}[0pt]{\@author}\hfill{}} & \\[3ex]
%       \raisebox{-3ex}[0pt]{\makebox[4.8cm][s]{学\hfill 号: }} & {\songti {}\raisebox{-3ex}[0pt]{\the\StudentNumber}\hfill{}} & \\[3ex]
%       \raisebox{-3ex}[0pt]{\makebox[4.8cm][s]{指\hfill 导\hfill 教\hfill 师\hfill 姓\hfill 名、\hfill 职\hfill 称: }} & {\songti
%       {}\raisebox{-3ex}[0pt]{\the\Csupervisor}\hfill{}} & \\[3ex] % \cline{2-2}
%       \raisebox{-3ex}[0pt]{\makebox[4.8cm][s]{专\hfill 业\hfill 类\hfill 别\hfill (领\hfill 域): }} & {\songti {}\raisebox{-3ex}[0pt]{\the\Cmajor}\hfill{}} &
%       \\[3ex]
%      \end{tabular}    
%     \else  
%     \begin{tabular}{cp{5.5cm}c}
%       \raisebox{-3ex}[0pt]{\makebox[4.8cm][s]{研\hfill 究\hfill 生\hfill 姓\hfill 名: }} & {\songti {}\raisebox{-3ex}[0pt]{\@author}\hfill{}} & \\[3ex]
%       \raisebox{-3ex}[0pt]{\makebox[4.8cm][s]{学\hfill 号: }} & {\songti {}\raisebox{-3ex}[0pt]{\the\StudentNumber}\hfill{}} & \\[3ex]
%       \raisebox{-3ex}[0pt]{\makebox[4.8cm][s]{指\hfill 导\hfill 教\hfill 师\hfill 姓\hfill 名、\hfill 职\hfill 称: }} & {\songti
%       {}\raisebox{-3ex}[0pt]{\the\Csupervisor}\hfill{}} & \\[3ex] % \cline{2-2}
%       \raisebox{-3ex}[0pt]{\makebox[4.8cm][s]{专\hfill 业\hfill 名\hfill 称: }} & {\songti {}\raisebox{-3ex}[0pt]{\the\Cmajor}\hfill{}} &
%       \\[3ex]
%       \raisebox{-3ex}[0pt]{\makebox[4.8cm][s]{研\hfill 究\hfill 方\hfill 向: }} & {\songti{}\raisebox{-3ex}[0pt]{\the\Cspeciality}\hfill{}} & \\[3ex]
%      \end{tabular}
%      \fi
%     }
%     \par
%     \vspace{25mm}
%     {
%       {\heiti \zihao{3} \@date \par}%
%     }
%   \end{center}%
%   \null
%   \iflib
%   \else
%   \newpage
%   \thispagestyle{empty}
%   \cleardoublepage
%   \fi
%   }

% %------------------------ Abstract & Keywords -------------------------
% \newcommand\cnkeywords[1]{ {\heiti\zihao{-4}\cnkeywordsname: }#1}
% \newcommand\cnkeywordsname{关键词}
% \newcommand\enkeywords[1]{ {\bfseries\zihao{-4}\enkeywordsname: }#1}
% \newcommand\enkeywordsname{Key words}

% \newenvironment{cnabstract}{
%     \newpage
%     \chapter[\cnabstractname]{\ziju{2}{\cnabstractname}}   %%%终于解决了书签乱码的问题 2005-04-12.huangzh
%     \songti\zihao{-4}
%     \@afterheading}
%     {\par}

% \newenvironment{enabstract}{
%     \newpage
%     \chapter{\enabstractname}
%     \songti\zihao{-4}
%     \@afterheading}
%     {\par}
% %%------------------------常用宏包-----------------------------------
% \RequirePackage{amsmath,amssymb}% AMSLaTeX宏包.
% \RequirePackage[amsmath,thmmarks]{ntheorem}
% \RequirePackage{graphicx}                 % 图形
% \RequirePackage{color,xcolor}             %支持彩色
% \RequirePackage{cite} % 参考文献引用, 得到形如 [3-7] 的样式.
% \RequirePackage{url}
% \RequirePackage{enumerate}
% %%----------------------- Theorems -----------------------------------
% \theoremstyle{plain}
% \theoremheaderfont{\heiti}
% \theorembodyfont{\songti} \theoremindent0em
% %\theorempreskip{0pt}
% %\theorempostskip{0pt}
% %\theoremseparator{\hspace{1em}}
% \theoremnumbering{arabic}
% %\theoremsymbol{} %定理结束时自动添加的标志
% \newtheorem{theorem}{\hspace{2em}定理}[section]
% \newtheorem{definition}{\hspace{2em}定义}[section]
% \newtheorem{lemma}{\hspace{2em}引理}[section]
% \newtheorem{corollary}{\hspace{2em}推论}[section]
% \newtheorem{proposition}{\hspace{2em}性质}[section]
% \newtheorem{example}{\hspace{2em}例}[section]
% \newtheorem{remark}{\hspace{2em}注}[section]

% \theoremstyle{nonumberplain}
% \theoremheaderfont{\heiti}
% \theorembodyfont{\normalfont \rm \songti}
% \theoremindent0em \theoremseparator{\hspace{1em}}
% \theoremsymbol{$\square$}
% \newtheorem{proof}{\hspace{2em}证明}

% %%%--- \upcite ----------------------------------------
% \newcommand{\upcite}[1]{\textsuperscript{\cite{#1}}}  %自定义新命令\upcite, 使参考文献引用以上标出现

%%%%%%%-------------------------------------------------
% %%%===  suppress extra inter-line spacing in \list environments
% \makeatletter \setlength\partopsep{0pt}
% \def\@listI{\leftmargin\leftmargini
%             \parsep 0pt
%             \topsep \parsep
%             \itemsep \parsep}
% \@listI
% \def\@listii {\leftmargin\leftmarginii
%               \labelwidth\leftmarginii
%               \advance\labelwidth-\labelsep
%               \parsep    \z@ \@plus\z@  \@minus\z@
%               \topsep    \parsep
%               \itemsep   \parsep}
% \def\@listiii{\leftmargin\leftmarginiii
%               \labelwidth\leftmarginiii
%               \advance\labelwidth-\labelsep
%               \parsep    \z@ \@plus\z@  \@minus\z@
%               \topsep    \parsep
%               \itemsep   \parsep}
% \def\@listiv{\leftmargin\leftmarginiv
%               \labelwidth\leftmarginiv
%               \advance\labelwidth-\labelsep
%               \parsep    \z@ \@plus\z@  \@minus\z@
%               \topsep    \parsep
%               \itemsep   \parsep}
% \def\@listv{\leftmargin\leftmarginv
%               \labelwidth\leftmarginv
%               \advance\labelwidth-\labelsep
%               \parsep    \z@ \@plus\z@  \@minus\z@
%               \topsep    \parsep
%               \itemsep   \parsep}
% \def\@listvi{\leftmargin\leftmarginvi
%               \labelwidth\leftmarginvi
%               \advance\labelwidth-\labelsep
%               \parsep    \z@ \@plus\z@  \@minus\z@
%               \topsep    \parsep
%               \itemsep   \parsep}
% %
% % Change default margins for \list environments
% \setlength\leftmargini   {2\ccwd} \setlength\leftmarginii
% {\leftmargini} \setlength\leftmarginiii {\leftmargini}
% \setlength\leftmarginiv  {\leftmargini} \setlength\leftmarginv
% {\ccwd} \setlength\leftmarginvi  {\ccwd} \setlength\leftmargin
% {\leftmargini} \setlength\labelsep      {.5\ccwd}
% \setlength\labelwidth    {\leftmargini}
% %
% \setlength\listparindent{2\ccwd}
% % Change \listparindent to 2\ccwd in \list
% \def\list#1#2{\ifnum \@listdepth >5\relax \@toodeep
%      \else \global\advance\@listdepth\@ne \fi
%   \rightmargin \z@ \listparindent2\ccwd \itemindent \z@
%   \csname @list\romannumeral\the\@listdepth\endcsname
%   \def\@itemlabel{#1}\let\makelabel\@mklab \@nmbrlistfalse #2\relax
%   \@trivlist
%   \parskip\parsep \parindent\listparindent
%   \advance\linewidth -\rightmargin \advance\linewidth -\leftmargin
%   \advance\@totalleftmargin \leftmargin
%   \parshape \@ne \@totalleftmargin \linewidth
%   \ignorespaces}


% %%%=== 图片路径=== %%%
% \graphicspath{{figures/}}        % 图片放在 figures 文件夹.