% !TeX encoding = UTF-8
% !TeX program = xelatex
% !TeX spellcheck = en_US

%********************************************
% whu-proposal: 武汉大学开题报告模版

% Github:https://github.com/whutug/whu-proposal
% Gitee:https://gitee.com/xkwxdyy/whu-proposal
% Update date: 2023-09-26
% Version: v0.8
% Author: Kangwei Xia, kangweixia_xdyy@163.com, School of Mathematics and Statistics, Wuhan University
% QQ group: 681965476
%********************************************


% \documentclass[type = bachelor]{whu-proposal}  % 本科生
% \documentclass[type = master]{whu-proposal}      % 硕士生
\documentclass[type = doctor]{whu-proposal}      % 博士生


% 个人信息
\ProposalSetup{
  title            = { 论文题目 } ,
  department       = { 数学与统计学院 } ,
  student_id       = { 2021202012345 } ,
  author           = { 作者姓名 } ,
  % 下面的内容只有【硕|博】需要填写
  major            = { 基础数学 } ,
  research_area    = { 算子理论 } , 
  supervisor       = { 导师姓名 } ,
  supervisor_title = { 教授 } ,
  year             = { 2023 },  % 年份不填写时默认为编译时的年份
  % month            = { 5 },     % 月份不填写时默认为编译时的月份
  % day              = { 21 },    % 日期不填写时默认为编译时的日期
}


% 载入所需宏包,下面的仅为示例文件中需要,实际使用时可根据需要增减

% 参考文献
\usepackage[
  backend      = biber,
  bibstyle     = gb7714-2015,
  citestyle    = gb7714-2015,
  sorting      = nyt,
  gbnamefmt    = givenahead,
  gbpunctin    = false
]{biblatex}
\addbibresource{whu-proposal.bib}  % 参考文献数据库



% 自定义命令,下面的仅为示例文件中需要,实际使用时可根据需要增减
\newcommand\tool{\texttt}
\newcommand\tn[1]{\texttt{\textbackslash#1}}
\newcommand\file{\nolinkurl}
\newcommand\pkg{\textsf}



\begin{document}

% 目录,不需要的话可去除
% \tableofcontents


% 正文内容
% !TeX root = ../whu-proposal-main.tex

\section{为什么会开发此模版}

武汉大学本科生的开题报告模版已经\href{https://github.com/whutug/whu-thesis}{由 whu-tug 开发},但是研究生的模版仍然只有 word 版本,对于数学系的同学来说,相较于 \LaTeX{} 来说,使用 word 进行排版数学公式并不是一件容易的事情。而且 \LaTeX{} 使用 bib 数据库进行参考文献的管理,可以很方便地实现参考文献的引用。

于是我基于《附件1:研究生学位论文开题报告登记表》\cite{研究生学位论文开题报告登记表} 开发了此模版,希望能够让同学们更方便、高效地完成开题报告的撰写。

\emph{注意}:开题报告上传系统的时候可能只能上传 \texttt{.doc} 文件,有以下解决办法:
\begin{enumerate}
  \item 将 \LaTeX{} 编译生成的 PDF 转为图片(一般的 PDF 编辑器都可以做到)
  \item 将图片插入 \texttt{.doc} 或 \texttt{.docx} 文件中保存
  \item 上传系统
\end{enumerate}
% !TeX root = ../../whuthesis-doc.tex

\begin{function}{\whusetup}
\begin{syntax}
  \verb|\whusetup| \marg{key-vals}
  \verb|\whusetup| \oarg{key path} \marg{key-vals}
  \verb|\whusetup| \{
  ~~\meta{key path_1} = \marg{key-vals_1} ,
  ~~\meta{key path_2} = \marg{key-vals_2} ,
  ~~...
  \}
\end{syntax}
键值设置命令。

\WhuThesis 的不同模块使用不同的 \meta{key path},一般情况下,这些模块会提供自己的键值设置接口,
为了使用 \cs{whusetup} 来设置这些键值,需要指定 \meta{key path}。
\end{function}

在本文档中,键的说明文字旁的表格中列出了键的完整写法,\meta{key path} 即为灰色的部分。
如键 \keyreflist[frame]{outer-sep, sep} 可以写成 
\begin{xample}
\whusetup[frame]{outer-sep=0pt, sep=20pt}
或 
\whusetup{ frame/outer-sep=0pt, frame/sep=20pt }
\stopxamplecode
\xamplecode\medskip
\end{xample}


\begin{function}{\whusetstyle}
\begin{syntax}
  \verb|\whusetstyle|   \oarg{key path} \marg{key} \marg{key-vals}
  \verb|\whusetstyle| * \oarg{key path} \marg{key} \marg{code}
\end{syntax}
自定义键。

带 \verb|*| 的可使用一个参数,它代表键传入的值。
\end{function}


\begin{function}[EXP]{\Replicate}
  \begin{syntax}
    \verb|\Replicate| \marg{num expr} \marg{code}
  \end{syntax}
  重复 \meta{code} \meta{num expr} 次。
\end{function}
% !TeX root = ../whu-proposal-main.tex

\section{可供用户参考的内容}

\subsection{2023年硕博开题报告的时间安排}

(以下来自班级群班长转发内容\cite{2024届硕博开题报告QQ通知})

各位同学,学位论文开题报告会预计在下学期9月开展,请大家做好开题准备:
\begin{enumerate}
  \item 开题报告会时间:2023年9中下旬(暂定),确切日期等各系通知;
  \item 开题前与导师及时、充分沟通论文选题及研究计划,确保选题合理,研究计划可行;开题需征得导师同意;
  \item 请将开题报告初稿于7月31日前发给导师审核,根据导师意见修改和完善;
  \item 硕士研究生开题报告不少于3千字,博士研究生开题报告不少于5千字,开题报告内容包括论文选题意义、国内外研究现状、论文研究计划、主要参考文献。行文主要篇幅用于写“论文研究计划”;
  \item 参加此次开题报告的学生名单见下表,如有遗漏,请告知教学办;
  \item 开题是进行学位论文撰写的必经环节,请大家认真对待。
\end{enumerate}


\subsection{数学与统计学院研究生学位论文开题报告管理实施细则}

下面内容来自 2023 年 5 月 18 日发布的《数学与统计学院研究生学位论文开题报告管理实施细则》\cite{数学与统计学院研究生学位论文开题报告管理实施细则}:

\subsubsection{开题报告的目的}

开题报告是学位论文研究的必经环节,是做好学位论文的重要前提,是为阐述、审核、确定学位论文选题及内容而举行的报告会,主要考察研究生学位论文选题是否恰当,国内外研究现状和趋势是否准确把握,研究计划是否科学可行,参考文献是否翔实充分等,旨在监督和保证硕士、博士学位论文质量。


\subsubsection{开题报告的完成时间}

开题报告前,研究生应结合学科专业培养目标、参与的课题研究、自身的专业基础、研究兴趣与专长,通过与导师的充分沟通与协商,选择和确定学位论文选题,广泛查阅文献资料,开展选题的探索性研究,并在选题得到合理评估的基础上撰写开题报告。

为保证学位论文写作及答辩质量,学术学位硕士学位论文开题报告应在硕士阶段第二学年第二学期前完成,专业学位硕士学位论文开题报告应在硕士阶段第一学年第二学期前完成。博士学位论文开题报告一般应在第二学年第一学期完成。开题应经导师审核同意,确定学位论文选题,在此基础上撰写开题报告。开题报告经专家评议通过后,方可进入学位论文撰写阶段。


\subsection{硕博开题报告的内容组成}


开题报告的内容一般由以下几部分组成:
\begin{enumerate}
  \item 论文选题的价值与意义;
  \item 与论文选题相关的国内外研究现状及趋势;
  \item 论文研究计划,包括
    \begin{enumerate}
      \item 研究目标、研究思路、研究方法、技术路线;
      \item 研究的重难点、可能的创新点与不足之处;
      \item 研究的主要内容及初步框架;
      \item 研究的进度安排;
    \end{enumerate}
  \item 主要参考文献。
\end{enumerate}

硕士研究生开题报告正文部分不得少于 3000 字,博士研究生开题报告正文部分不得少于 5000 字。\emph{其中第三部分“论文研究计划”应作为开题报告的重点阐述内容。}


\subsubsection{开题报告专家委员会的组成}

开题报告会一般由研究生所在学科专业集中统一组织,硕士学位论文开题报告由导师及相关学科专家组成 3-5 人考核小组。其中,专业学位研究生开题报告评议小组成员中至少有 1 名行业领域专家或校外导师。博士学位论文开题报告由导师及相关学科专家组成不少于 5 人的考核小组。

硕士研究生开题报告评议小组成员应有副高及以上职称,博士研究生开题报告评议小组成员应有正高职称或具有博士生指导资格。


\subsubsection{开题报告会的程序}


\begin{enumerate}
  \item 研究生教学管理办公室于开题报告会前两个月左右提醒需参加开题报告的研究生及其导师做好开题准备。
  \item 开题报告申请需征得导师同意。研究生将开题报告初稿于开题报告会前一个月请导师审核,根据导师审核意见对开题报告进行修改和完善。开题报告完成后,请导师于开题报告会前在《武汉大学研究生学位论文开题报告登记表》\cite{研究生学位论文开题报告登记表} 第 2 页“指导教师审核意见”栏填写导师意见并签名。
  \item 各系或各学科组织专家形成开题报告评议小组,每个评议小组安排一名小组秘书负责记录。
  \item 研究生以PPT形式作开题报告汇报,并提前准备纸质开题报告供评议专家查阅,硕士研究生汇报时间一般不少于15分钟,博士研究生汇报时间不少于20分钟。
  \item 评议小组成员对开题报告提出意见和建议。
  \item 评议小组做出开题报告的决议。
\end{enumerate}


\subsubsection{开题报告的决议}

开题报告的决议由评议小组成员集体商议后做出,决议分为“通过”、“不通过”两种。

开题报告“通过”者,可进入学位论文研究写作阶段;开题报告“不通过”者,须重新准备开题报告,由研究生本人向学院研究生教学管理办公室书面申请后可参加下一次开题报告会。

开题报告通过后,原则上不得更改论文选题。确因实际研究需要对论文选题进行微调的,在核心研究工作和论文主框架不发生大的调整的情况下,导师可指导修改或调整论文题目。若论文核心研究工作和论文主框架发生了大幅度调整,尤其是论文选题方向发生了变化的情况下,需要重新申请进行开题报告评审。


\subsubsection{开题报告的存档与管理}

(一)开题报告会结束后,评议小组秘书将所在小组的学生开题报告评议结果记录在《研究生学位论文开题报告评议结果汇总表》\cite{研究生学位论文开题报告登记表} 中,汇总表由评议小组秘书交至研究生教学管理办公室存档备案至该届研究生毕业。

(二)研究生于开题报告会结束后一周内,在研究生综合管理信息系统中录入开题报告结果,上传开题报告登记表扫描件及完整版的开题报告PDF文件作为附件,资料不全者将无法通过审核。只有开题报告结果为“通过”状态者,才能在系统中申请毕业和申请学位。

(三)研究生将纸质版开题报告登记表胶装成册于开题报告会结束后一个月之内,以班级为单位交至研究生教学管理办公室保存,在研究生毕业时存入研究生学位档案袋中。



% 参考文献

% \printbibliography  % “参考文献”一节标题不编号
\printbibliography[heading = bibnumbered]  % “参考文献”一节标题编号


\end{document}