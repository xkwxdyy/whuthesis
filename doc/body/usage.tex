% !TeX root = ../../whuthesis-doc.tex

\begin{function}{\whusetup}
\begin{syntax}
  \verb|\whusetup| \marg{key-vals}
  \verb|\whusetup| \oarg{key path} \marg{key-vals}
  \verb|\whusetup| \{
  ~~\meta{key path_1} = \marg{key-vals_1} ,
  ~~\meta{key path_2} = \marg{key-vals_2} ,
  ~~...
  \}
\end{syntax}
键值设置命令。

\WhuThesis 的不同模块使用不同的 \meta{key path},一般情况下,这些模块会提供自己的键值设置接口,
为了使用 \cs{whusetup} 来设置这些键值,需要指定 \meta{key path}。
\end{function}

在本文档中,键的说明文字旁的表格中列出了键的完整写法,\meta{key path} 即为灰色的部分。
如键 \keyreflist[frame]{outer-sep, sep} 可以写成 
\begin{xample}
\whusetup[frame]{outer-sep=0pt, sep=20pt}
或 
\whusetup{ frame/outer-sep=0pt, frame/sep=20pt }
\stopxamplecode
\xamplecode\medskip
\end{xample}


\begin{function}{\whusetstyle}
\begin{syntax}
  \verb|\whusetstyle|   \oarg{key path} \marg{key} \marg{key-vals}
  \verb|\whusetstyle| * \oarg{key path} \marg{key} \marg{code}
\end{syntax}
自定义键。

带 \verb|*| 的可使用一个参数,它代表键传入的值。
\end{function}


\begin{function}[EXP]{\Replicate}
  \begin{syntax}
    \verb|\Replicate| \marg{num expr} \marg{code}
  \end{syntax}
  重复 \meta{code} \meta{num expr} 次。
\end{function}