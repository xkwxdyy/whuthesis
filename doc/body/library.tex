% !TeX root = ../../whuthesis-doc.tex

\chapter{库}

\cls{whuthesis} 并不是第一个采用模块化处理思想的 \LaTeX{} 论文模板,早在 \cls{fduthesis} 就使用了 \pkg{xtemplate} 宏包

\section{\whulibrary{proposal} 库:开题报告}


\section{\whulibrary{bachelor} 库:本科}


\section{\whulibrary{master} 库:硕士}


\section{\whulibrary{doctor} 库:博士}


\section{学院配置}

\subsection{\whulibrary{math} 库:数学与统计学院}

本库调用了与数统学院黄正华老师模板基本相同的配置,但仍有一些细节进行了调整:

修改:
\begin{enumerate}
  \item 更新了武汉大学校徽\cite{whulogo}
  \item 去掉了摘要、致谢等章节标题样式中的 \cs{ziju} 设置(因为经测试比对,是否添加 \cs{ziju} 设置对排版效果没有影响)
  \item 专业硕士封面中“专业类别(领域)”的括号改为中文括号
\end{enumerate}

优化:
\begin{enumerate}
  \item 封面采用 \pkg{tikz} 的处理,避免标题长度影响封面的排版
  \item 基于 \pkg{amsthm} 宏包里的 \env{proof} 环境进行格式修改,可以使用 \cs{qedhere} 命令改变证明结束符的位置(黄正华老师模板中无法使用 \cs{qedhere} 命令)
  \item 解决了黄正华老师模板中未完全解决的消除空白页问题(\verb|type = for-library| 可以去除所有空白页)
\end{enumerate}


\subsubsection{\whulibrary{math.bachelor} 库:数学与统计学院\&本科}

\verb|library={bachelor, math}| 和 \verb|library={math.bachelor}| 的效果相同,会调用黄正华老师的本科模板设置,但注意前者的 \whulibrary{bachelor} 库和 \whulibrary{math} 库的顺序不能颠倒,所以推荐直接使用 \verb|library={math.bachelor}|。

如果只载入了 \whulibrary{math} 库而没有载入 \whulibrary{bachelor}、\whulibrary{master} 和 \whulibrary{doctor} 的任意一个库,则默认调用 \whulibrary{math.bachelor} 库。


\subsubsection{\whulibrary{math.master} 库:数学与统计学院\&硕士}


\verb|library={master, math}| 和 \verb|library={math.master}| 的效果相同,会调用黄正华老师的硕士模板设置,但注意前者的 \whulibrary{master} 库和 \whulibrary{math} 库的顺序不能颠倒,所以推荐直接使用 \verb|library={math.master}|。


\subsubsection{\whulibrary{math.doctor} 库:数学与统计学院\&博士}


\verb|library={doctor, math}| 和 \verb|library={math.doctor}| 的效果相同,会调用黄正华老师的博士模板设置,但注意前者的 \whulibrary{doctor} 库和 \whulibrary{math} 库的顺序不能颠倒,所以推荐直接使用 \verb|library={math.doctor}|。

\subsection{\whulibrary{computer} 库:计算机学院}


\subsection{\whulibrary{cs} 库:国家网络安全学院}