% !TeX root = ../../whuthesis-doc.tex

\PassOptionsToPackage{silent}{xeCJK}
\usepackage{ctex}
\usepackage{fontawesome5}

\usepackage[library={doc,box,bnf,ref,counter,pgf}]{whu}
\SetKeys[updatemarks]{multicol}
\newcommand{\hook}{\cmd[module=hook point,type=hook point]}
\newcommand{\colna}{\key[module=color name,type=color name]}
\usepackage{marginnote}

\makeatletter
\newcommand{\sdanger}[1][1]{\par\medskip\noindent\@@line{\hss\Replicate{#1}{\textdbend}\hss}\par}
\newcommand{\mdanger}[1][1]{\marginnote{\Replicate{#1}{\textdbend}}}
\makeatother

\newcommand{\pkgdoc}[1]{\pkg{#1} 宏包文档}
\newcommand*{\TODO}{\textcolor{red!90!black}{\bfseries[TODO]}}
\newcommand\UNEXPANDEDRESULT{最终结果使用 \tn{unexpanded} (\cs{exp_not:n})包裹起来。}

\setuplayout*[balance]{hmargin=1.7cm,top=2.3cm,bottom=2.5cm,
  hfoffset=0pt,nomarginpar,
  columnsep=35pt,headsep=10pt,footskip=30pt,}
\setuplayout[main]{paper=a4,
  marginparsep=10pt,marginparwidth=15\ccwd,
  textwidth=35\ccwd,inner=1.7cm,top=2.3cm,bottom=2.5cm,
  headsep=10pt,footskip=30pt,
  hfoffset={[OR,EL]168.1pt},%marking,%showframe,
}

\usepackage{graphicx}
\graphicspath{{.}{./whu-aux/}{./whudoc-aux}{./pic}}
\usepackage{xcolor}

% region math & fonts
\usepackage{amsmath, amsfonts}
\usepackage{unicode-math}
% \setmainfont{texgyrepagella}[
%   Extension      = .otf,
%   UprightFont    = *-regular,
%   BoldFont       = *-bold,
%   ItalicFont     = *-italic,
%   BoldItalicFont = *-bolditalic]
\setmainfont{TeXGyreTermesX}[
  Extension      = .otf,
  UprightFont    = *-Regular,
  BoldFont       = *-Bold,
  ItalicFont     = *-Italic,
  BoldItalicFont = *-BoldItalic,
  SlantedFont    = *-Slanted,
  BoldSlantedFont= *-BoldSlanted]
\setsansfont{texgyreheros}[
  Extension      = .otf,
  UprightFont    = *-regular,
  BoldFont       = *-bold,
  ItalicFont     = *-italic,
  BoldItalicFont = *-bolditalic]
\setmonofont{cmun}[
  Extension      = .otf,
  UprightFont    = *btl,
  BoldFont       = *tb,
  ItalicFont     = *bto,
  BoldItalicFont = *tx,
  HyphenChar     = None]
\setmathfont{texgyrepagella-math.otf}
%endregion

\usepackage{lineno}
\usepackage{caption}
% \expandafter\let\csname caption@ifoddpage\endcsname\IfPageOdd
\usepackage{floatrow}
\usepackage{tikz}
\usetikzlibrary{shadings}
\usepackage[many]{tcolorbox}
\usepackage{listings}
% \usepackage{siunitx}
\usepackage{adjustbox}
\usepackage[strict]{changepage}
\renewcommand{\cplabel}{@}
\usepackage{array,booktabs,tabularx,makecell}
\usepackage{tabularray}

\floatsetup{captionskip=5pt,facing=yes}

\ExplSyntaxOn
\msg_redirect_name:nnn { tabularray } { table-width-too-small } { log }
\ExplSyntaxOff

\usepackage[colorlinks]{hyperref}
\hypersetup{
  pdfauthor={夏大鱼羊},
  pdftitle=\WhuThesis 手册
}
\usepackage{nameref,varioref,cleveref}

\usepackage[numbered,open,openlevel=1]{bookmark}

\usepackage{glossaries}

\usepackage{zhlipsum}

\DeclareNewFloatType{example}{fileext=example,name=例}

\newfloatcommand{mtabbox}{table}[\setcaptiontype{table}\captop][\FBwidth]

\enablecombinedlist 

\newsavebox\WaterMarkBox
\sbox{\WaterMarkBox}{\rotatebox{45}{\color{gray!30}\fontsize{100}{0}\sffamily \WhuThesis}}
\background+[./watermark]{\copy\WaterMarkBox}

%region title setting
\makeatletter
\@secpenalty=-\@m 
%% 空 mark 实际上保存为了 \prg_do_nothing:,这里并不直接检测,而利用一个空的 mark
\NewMarkClass{totalempty}
\InsertMark{totalempty}{}
\NewMarkClass{chapter/head}
\NewMarkClass{section/head}
\long\def\chaptermark#1{%
  \InsertMark{chapter/head}{\protect\hyperlink{\@currentHref}{#1}}%
  \InsertMark{section/head}{}}
\long\def\sectionmark#1{%
  \InsertMark{section/head}{\protect\hyperlink{\@currentHref}{#1}}}
\def\@chaptosec{\;\texttt{>\kern-.1em >}\;}
\def\@splitrange{\ \texttt{=\kern-.1em =\kern-.2em >}\ }
\def\head@ifempty#1{\ifthenelse{\equal{#1}{}\OR\equal{#1}{\TopMark{totalempty}}}}
\def\head@hifeq#1{\IfMarksEqualTF{#1/head}}
\def\head@ifeq#1#2{\ifthenelse{\equal{#1}{#2}}}
\def\head@ct{\TopMark{chapter/head}} \def\head@st{\TopMark{section/head}}
\def\head@cf{\FirstMark{chapter/head}} \def\head@sf{\FirstMark{section/head}}
\def\head@cl{\LastMark{chapter/head}} \def\head@sl{\LastMark{section/head}}
\def\head@fc#1{{\bfseries #1}} \def\head@fs#1{{#1}}
\newcommand\marked@title{%
  \head@hifeq{chapter}{first}{last}{%
    \head@ifempty\head@cf{}
      {\head@ifempty\head@ct
        {\head@fc\head@cf\head@ifempty\head@sl{}{\@chaptosec
          \head@ifeq\head@sf\head@sl{\head@fs\head@sf}{\head@fs\head@sf
            \head@ifempty\head@sl{}{\@splitrange\head@fs\head@sl}}}}
        {\head@ifeq\head@ct\head@cf
          {\head@fc\head@ct\head@ifempty\head@sf{}
            {\@chaptosec\head@ifeq\head@sf\head@sl
              {\head@fs\head@sf}{\head@fs\head@sf\@chaptosec\head@fs\head@sl}}}
          {\head@fc\head@ct\head@ifempty\head@st{}{\@chaptosec\head@fs\head@st}\@splitrange
            \head@fc\head@cl\head@ifempty\head@sl{}{\@chaptosec\head@fs\head@sl}}}}%
  }{\head@fc\head@cf\@splitrange\head@fc\head@cl}}
\setuptitle[chapter]{pagestyle=fancy, 
  fixskip, break=\needspace{100pt}, 
  beforeskip=30pt, afterskip=25pt, format=\zihao{-2}\bfseries\centering,}
\setuptitle[section]{fixskip, break=\addpenalty{-5000},
  name={\texorpdfstring{\S~}{\S}},
  beforeskip=20pt plus 5pt minus 5pt, afterskip=15pt plus 2pt minus 2pt,
  format=\zihao{-3}\bfseries\raggedright,}
\setuptitle[subsection]{fixskip,
  beforeskip=10pt plus 3pt minus 3pt,
  afterskip=10pt plus 3pt minus 3pt,
  format=\zihao{-4}\bfseries\raggedright,}

\providecommand\headlink@warp[1]{#1}
\setpagestyle{plain}[totalempty]{}
\setpagestyle*{fancy}[totalempty]{
  \sethead [ol,er] {\WhuThesis 手册}
  \sethead [or,el] {\headlink@warp\marked@title}
  % \setfoot [or,el] {\texttt{Longaster@163.com}}
  \setfoot [ol,er] {第\thepage 页}
  \setheadrulewidth {1pt}
}
\makeatother
%endregion

\makeatletter
\protected\def\this@doc@externallink{\raisebox{-.8ex}[0pt][0pt]{\hi{\faIcon{external-link-alt}}}}
\protected\def\this@doc@linkpage{\raisebox{.8ex}[0pt][0pt]{\lo{P.\labelinfo{page}{\whu@doc@thelabel}}}}
\protected\def\this@doc@linkinfo{\lohi{\,\rmfamily\this@doc@linkpage}{\,\this@doc@externallink}}
\def\whu@doc@basic@format{\ifx\whu@doc@thelabel\@empty \whu@doc@thetext
  \else \hyperref[\whu@doc@thelabel]{\whu@doc@thetext\this@doc@linkinfo}\!\fi}
\def\whu@doc@cs@format#1{\hypersetup{linkcolor=whu/color/doc cs}\whu@doc@basic@format}
\let\whu@doc@cmd@format\whu@doc@cs@format
\let\whu@doc@tn@format\whu@doc@cs@format
\def\whu@doc@key@format#1{\hypersetup{linkcolor=whu/color/doc key}\whu@doc@basic@format}
\def\whu@doc@csref@format#1{\whu@doc@basic@format}
\def\whu@doc@envref@format#1{\whu@doc@basic@format}
\def\whu@doc@keyref@format#1{\whu@doc@basic@format}
\setfontface\whu@doc@itfont{texgyrepagella-italic}[Extension=.otf,BoldFont=texgyrepagella-bolditalic]
\makeatother
\definecolor{whu/color/doc cs}{RGB}{180,116,107}
\definecolor{whu/color/doc env}{RGB}{216,156,122}
\definecolor{whu/color/doc key}{RGB}{138,149,169}
\newindextype[auto=true,filename=\jobname.idx,heading*={\section}]{\empty}
\setupindex[\empty,docchange]{auto=false}


%region aux env
\makeatletter
\lstdefinestyle{xamplestyle}{language={[LaTeX]TeX},
  basicstyle=\small\linespread{1.1}\ttfamily,
  aboveskip=\smallskipamount,belowskip=-\medskipamount,
  % aboveskip={0\p@ \@plus 6\p@}, belowskip={0\p@ \@plus 6\p@},
  columns=fullflexible, keepspaces=true,
  breaklines=true, breakatwhitespace=true, 
  breakindent=0pt, postbreak={\hb@xt@1.5em{\hss{\color{gray}$\hookrightarrow$}\hss}},
  extendedchars=true, nolol,
  numberstyle=\tiny,numbersep=8pt,
  commentstyle=\color{green!55!black}}
\definebufferpair [ __process_line=standard-not-nest,
  save-mode=write, write=\jobname.exambuff, blank=space] 
  \startxamplecode \stopxamplecode {} 
  {\edef\xamplecode{\noexpand\lstinputlisting[style=xamplestyle,\unexpanded\expandafter{\xampleOPTlst}]{\jobname.exambuff}}%
    \def\xcaption{\setcaptiontype{example}\caption}%
    \xample@hango\begin{longfbox}[]\xample@hangi}
\protected\def\xample@hango{%
  \par\refstepcounter{example}%
  \edef\xample@test{\noexpand\ifnum\the\c@example>\c@example 
    \global\advance\c@example\@ne\noexpand\fi}}
\protected\def\xample@hangi{%
  \setbox\z@\hb@xt@\textwidth{\hss{\color{red!80!black}\bfseries 例\theexample}}%
  \global\advance\c@example -\@ne
  \vspace*{-\dimexpr\option{/fbox/padding-top}+\parskip}\par
  \edef\xample@tmp{\vskip-\the\dimexpr\ht\z@+\dp\z@+.3333em+\parskip\relax\par}%
  \box\z@ \xample@tmp}
\protected\def\xampleline{\noindent \kern-\dimexpr\option{/fbox/padding-left}\relax
  \cleaders\hb@xt@.2em{\hss.\hss}\hfill 
  \kern\dimexpr-\option{/fbox/padding-right}\relax \par}
\def\xampletext{\par\input{\jobname.exambuff}}
\def\xampleprint{\xamplecode \xampleline \xampletext}
\NewDocumentEnvironment {xample} { O{} O{} }
  {\penalty-1000 \def\xampleOPTlst{#1}%
    \fboxset{breakable=true,#2}%
    \expandafter\expandafter\expandafter\startxamplecode
      \expandafter\string\@firstofone}
  {\end{longfbox}\xample@test\goodbreak}

\lstdefinelanguage[BNF]{TeX}[common]{TeX}{
  texcs=[1]{BNFItem}, texcs=[2]{BNFN}, texcs=[3]{BNFT}, 
  texcs=[4]{BNFI,is}, texcs=[5]{BNFO,alt},
  moredelim=[s][{\color{blue}}]{<}{>},
  moredelim=[s][{\color{red!70}}]{"}{"},
  literate={{:}{{\bfseries\color{green!50!black}:}}1 {::=}{{\bfseries\color{green!50!black}::=}}3 {|}{{\bfseries\color{cyan}\string|}}1},
  texcsstyle={[1]{\color{purple}}}, texcsstyle={[2]{\color{blue}}}, 
  texcsstyle={[3]{\color{red!70}}}, texcsstyle={[4]{\bfseries\color{green!50!black}}},
  texcsstyle={[5]{\bfseries\color{cyan}}}}

\protected\def\normalsize{%
  \@setfontsize \normalsize {10.53937}{12.64725}%
  \abovedisplayskip 1\p@ \@plus 4\p@ \@minus 2\p@ 
  \abovedisplayshortskip \z@ \@plus 2\p@ 
  \belowdisplayskip \abovedisplayskip 
  \belowdisplayshortskip \abovedisplayshortskip
  \let \@listi \@listI
}
\makeatother
%endregion

\ExplSyntaxOn
\cs_new:Npn \UseList { \whu_tl_use:nn }
\cs_new:Npn \UseClist { \clist_use:nn }
\ExplSyntaxOff

\crefformat{figure}{#2图#1#3}
\crefformat{table}{#2表#1#3}
\crefformat{example}{#2例#1#3}
\crefformat{part}{#2第\zhnumber{#1}部分#3}
\crefformat{chapter}{#2第\zhnumber{#1}章#3}
\crefformat{section}{第 #2#1#3 节}
\crefformat{subsection}{第 #2#1#3 小节}

\setlist{nosep}
\setlist[itemize,1]{label=\textbullet}
\setlist[itemize,2]{label=$\smwhtcircle$}

\def\nofuncskip{\par\vskip-\bigskipamount\vskip\parskip\par}

\raggedbottom \hfuzz=2.5pt \vfuzz=10pt 

\title{\WhuThesis 手册}
\author{夏大鱼羊}
\date{\zhtoday\quad v\UseName{whu@versi@n}}